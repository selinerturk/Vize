% Options for packages loaded elsewhere
\PassOptionsToPackage{unicode}{hyperref}
\PassOptionsToPackage{hyphens}{url}
\PassOptionsToPackage{dvipsnames,svgnames,x11names}{xcolor}
%
\documentclass[
  12pt,
]{article}
\usepackage{amsmath,amssymb}
\usepackage{lmodern}
\usepackage{iftex}
\ifPDFTeX
  \usepackage[T1]{fontenc}
  \usepackage[utf8]{inputenc}
  \usepackage{textcomp} % provide euro and other symbols
\else % if luatex or xetex
  \usepackage{unicode-math}
  \defaultfontfeatures{Scale=MatchLowercase}
  \defaultfontfeatures[\rmfamily]{Ligatures=TeX,Scale=1}
\fi
% Use upquote if available, for straight quotes in verbatim environments
\IfFileExists{upquote.sty}{\usepackage{upquote}}{}
\IfFileExists{microtype.sty}{% use microtype if available
  \usepackage[]{microtype}
  \UseMicrotypeSet[protrusion]{basicmath} % disable protrusion for tt fonts
}{}
\makeatletter
\@ifundefined{KOMAClassName}{% if non-KOMA class
  \IfFileExists{parskip.sty}{%
    \usepackage{parskip}
  }{% else
    \setlength{\parindent}{0pt}
    \setlength{\parskip}{6pt plus 2pt minus 1pt}}
}{% if KOMA class
  \KOMAoptions{parskip=half}}
\makeatother
\usepackage{xcolor}
\usepackage[margin=1in]{geometry}
\usepackage{longtable,booktabs,array}
\usepackage{calc} % for calculating minipage widths
% Correct order of tables after \paragraph or \subparagraph
\usepackage{etoolbox}
\makeatletter
\patchcmd\longtable{\par}{\if@noskipsec\mbox{}\fi\par}{}{}
\makeatother
% Allow footnotes in longtable head/foot
\IfFileExists{footnotehyper.sty}{\usepackage{footnotehyper}}{\usepackage{footnote}}
\makesavenoteenv{longtable}
\usepackage{graphicx}
\makeatletter
\def\maxwidth{\ifdim\Gin@nat@width>\linewidth\linewidth\else\Gin@nat@width\fi}
\def\maxheight{\ifdim\Gin@nat@height>\textheight\textheight\else\Gin@nat@height\fi}
\makeatother
% Scale images if necessary, so that they will not overflow the page
% margins by default, and it is still possible to overwrite the defaults
% using explicit options in \includegraphics[width, height, ...]{}
\setkeys{Gin}{width=\maxwidth,height=\maxheight,keepaspectratio}
% Set default figure placement to htbp
\makeatletter
\def\fps@figure{htbp}
\makeatother
\setlength{\emergencystretch}{3em} % prevent overfull lines
\providecommand{\tightlist}{%
  \setlength{\itemsep}{0pt}\setlength{\parskip}{0pt}}
\setcounter{secnumdepth}{5}
\newlength{\cslhangindent}
\setlength{\cslhangindent}{1.5em}
\newlength{\csllabelwidth}
\setlength{\csllabelwidth}{3em}
\newlength{\cslentryspacingunit} % times entry-spacing
\setlength{\cslentryspacingunit}{\parskip}
\newenvironment{CSLReferences}[2] % #1 hanging-ident, #2 entry spacing
 {% don't indent paragraphs
  \setlength{\parindent}{0pt}
  % turn on hanging indent if param 1 is 1
  \ifodd #1
  \let\oldpar\par
  \def\par{\hangindent=\cslhangindent\oldpar}
  \fi
  % set entry spacing
  \setlength{\parskip}{#2\cslentryspacingunit}
 }%
 {}
\usepackage{calc}
\newcommand{\CSLBlock}[1]{#1\hfill\break}
\newcommand{\CSLLeftMargin}[1]{\parbox[t]{\csllabelwidth}{#1}}
\newcommand{\CSLRightInline}[1]{\parbox[t]{\linewidth - \csllabelwidth}{#1}\break}
\newcommand{\CSLIndent}[1]{\hspace{\cslhangindent}#1}
\usepackage{polyglossia}
\setmainlanguage{turkish}
\usepackage{booktabs}
\usepackage{caption}
\captionsetup[table]{skip=10pt}
\ifLuaTeX
  \usepackage{selnolig}  % disable illegal ligatures
\fi
\IfFileExists{bookmark.sty}{\usepackage{bookmark}}{\usepackage{hyperref}}
\IfFileExists{xurl.sty}{\usepackage{xurl}}{} % add URL line breaks if available
\urlstyle{same} % disable monospaced font for URLs
\hypersetup{
  pdftitle={Türkiye'de Alınan Verilen Göçlerin İllere Dağılımı-Cinsiyet-Yaş-Uyruk İncelemesi},
  pdfauthor={Selin Ertürk},
  colorlinks=true,
  linkcolor={Maroon},
  filecolor={Maroon},
  citecolor={Blue},
  urlcolor={blue},
  pdfcreator={LaTeX via pandoc}}

\title{Türkiye'de Alınan Verilen Göçlerin İllere Dağılımı-Cinsiyet-Yaş-Uyruk İncelemesi}
\author{Selin Ertürk\footnote{21080351, \href{https://github.com/selinerturk/Vize}{Github Repo}}}
\date{}

\begin{document}
\maketitle

\hypertarget{giriux15f}{%
\section{Giriş}\label{giriux15f}}

Göç insanların doğal, ekonomik, sosyal ve siyasal nedenlerden dolayı sürekli yaşadığı yerlerden başka yerlere toplu olarak veya bireysel olarak yerleşmeleri olayına denir.Bu yerleşim ülke içinde veya ülkeler arası olabilir. Her ikisinin ana sebebi de mevcut şartlardan daha iyi olan şartlara sahip bir hayata geçmektir. Bu şartlar detaylı bir bakış açısıyla değerlendirildiğinde ülkenin durumu hakkında önemli bilgileri bize sunar. Dışa göç verme oranları artan bir ülkenin şartlarında olumsuz gelişen durumlar olduğunu söylemek yanlış olmaz. Aynı şekilde dışardan göç alan bir ülke içinde diğer ülkelerden daha gelişmiş şartlara sahip olduğunu söyleyebiliriz. Göç genel olarak olumlu olumsuz olarak yorum yapabilirken arka planda yarattığı etkilerin de göz önünde bulundurulması gereken bir kavramdır. Çünkü göçün yarattığı bu mekansal değişim aynı zamanda birey, toplum ve milletlerin anlam ve değer dünyasının değişimi ve dönüşümünü de ifade etmektedir. Bu değişim ve dönüşümle beraber terk edilen ve yerleşilen yeni mekânla birlikte yeni mekânlardaki insanlar, değerler, kültürler ve insan ile ilgili ne varsa etkilenir.Ülkenin verdiği ve aldığı göçler olumlu olumsuz bir çok yansımayla karşılacaktır.

\hypertarget{uxe7alux131ux15fmanux131n-amacux131}{%
\subsection{Çalışmanın Amacı}\label{uxe7alux131ux15fmanux131n-amacux131}}

Veri setim 2018-2019 yılları arsında Türkiye'de görülen dış göçlerin illere göre dağılımını,yaş aralıkları ve cinsiyet ve bireylerin Türk uyruklu ve yabancı uyruklu olması durumunu içermektedir.Çalışmamın amacı ;

\begin{enumerate}
\def\labelenumi{\arabic{enumi}.}
\tightlist
\item
  Bu iki yıl esas alınarak dış göçün en çok etki ettiği iller hangileridir ve bu etkinin sebep sonuçları nasıl değerlendirmelidir? 2.Göçlerde etkili olan yaş grupları nelerdir ve bununla ilgili alınması gereken önlemler nelerdir?
  3.Göçlerde etkili olan cinsiyetler ve toplumdaki yeri nedir?
  4.Alınan ve verilen göçlerle birlikte ortaya çıkan nüfus değişikliği nasıl değerlendirilmelidir?
\end{enumerate}

Analizde ortaya koymak istediğim, tüm bu değişkenlerle birlikte göçün ortaya çıkardığı sonuçları temellere dayandırarak nasıl değerlendirmek gerekir,ne gibi politikalar izlemek gerekir sorularını yantlamaktır.

\hypertarget{literatuxfcr}{%
\subsection{Literatür}\label{literatuxfcr}}

\#\#\#Kadınlar

Mevcut göç akımlarıyla ilgili yenilikler, kadın göçmenlerin giderek görünür olması, kadınların göç hareketlerine artık bir aile bireyi olarak değil, tek başına hareket eden göçmenler olarak katılmaları ve sınırlar ötesine hareket etmeleridir. Kadınların göç süreçlerinin önemli aktörleri haline gelmesi hem göçün feminizasyonu hem de buna paralel olarak meydana gelen işgücü piyasasının feminizasyonu ile birlikte ortaya çıkmıştır..Son yıllarda bilimde cinsiyet eşitliğine yönelik önemli ilerlemeler kaydedilmesine rağmen, kadın araştırmacılar akademik işgücü piyasasında önemli engellerle karşılaşmaya devam ediyor. Uluslararası hareketlilik, bilim insanlarının profesyonel ağlarını genişletmeleri için bir strateji olarak giderek daha fazla kabul görmektedir ve bu, akademik kariyerlerdeki cinsiyet farkını daraltmaya yardımcı olabilir.Akademide cinsiyet eşitliğini teşvik eden çok çeşitli programlar başlatılmış olsa da, cinsiyet eşitsizlikleri akademi ve bilimlerin neredeyse tüm yönlerinde devam etmektedir.{[}zhao2023gender{]} Bu iki durumdan ortak çıkacak durum ise göçlerde kadın nüfusunun artmasına karşılık bu artışın yönetiminin eksik kalınan bir durum olmasıdır.Bu sorunun çözümü için atılacak ilk adım ülke genelinde kadın göçmenlerin yoğun olduğu şehirlerin ve eğitim-iş alanlarında etkin yaş gruplarına yönelik bir tespitin devamında eğitimlerini daha nitelikli bir hale getirecek çalışmalar yapılmalı ve bununla beraber ülkede hem kadın haklarına hem eğitime bir ivme kazandırılmalıdır.

\#\#\#Aile yapısı ve uyum

Ailelerin parçalanma sebeplerinden birisi de iç ve dış göçlerdir. Göç olgusu, tarih boyunca toplumların ve bireylerin yaşamını ekonomik, siyasal, kültürel, sosyal ve psikolojik olarak çok yönlü ve karmaşık bir şekilde etkilemiştir. Özellikle imkânların yeterli olmadığı yerlerde yaşayan bireyler, yurt içinde ya da yurt dışına göç etmek zorunda kalmışlardır. Ebeveynlerin işgücü amaçlı olarak başka yerlere göç etmesi, ailenin parçalanması ve aile bireylerinin birbirinden ayrı düşmesi ile sonuçlanabilmektedir. Anne ya da babanın uzun zaman aileden ayrı kalması, aile içinde meydana gelen hastalıklar ve ölümler, kişilerin sosyal, duygusal durumlarına ve bireyler arası iletişime zarar vererek aile sağlığını olumsuz şekilde etkilemektedir.{[}adem2014dics{]}Bu sorunlar beraberinde bireylerin iç dünyasında sıkıntılara yol açmakta ve sorunlar toplumda huzursuzlukla sonuçlanmaktadır.Asıl sebep bireylerin hayat şartlarında yaşadığı değişikliğe karşı yaşadığu uyumsuzluktan kaynaklanmaktadır.(\protect\hyperlink{ref-csimcsek2018turkiye}{ŞİMŞEK ve KULA, 2018}) Huzursuzluğun önüne geçmek ise göç konusunda önemli amaçlardan biri olmalıdır.Toplumun genelinde huzur ortamının sağlanması aile yapısının sağlam şekilde devam etmesini sağlamaktır.Bunun için öncelik dışa verilen göçlerin sebeplerinin çözümüne kavuşturularak ailelerin şartlarının ülke içinde bütünlük dahilinde düzeltirek dağılmanın önüne geçmektir.ikinci olarak ise dış göç ile ülkeye gelmiş olan yabancı uyruklu kişilerin ülke şartalarına zıt davranışlar sergileyerek huzursuzluk yaratma durumlarının önüne geçerek gelen dış göçün etkili olduğu illere kurulan dil-kültür gibi konuların bireylere aktarılmasını sağlayavak kurumlar ile baraber hem gelen bireylerin uyumsuzlukla daha iyi mücadele etmesi sağlanmalı hemde ülkenin genel düzeninde meydana gelecek bozulmaların önüne geçilmiş olunmalıdır.

\#\#\#Sağlık
Nüfus artışı, günümüzün en önemli çevre sorunlarından birisidir. İçme ve kullanma suyu, gürültü, hava kirliliği, radyasyon ve atıklar gibi pek çok çevre sağlığı konusunda nüfus artışıyla birlikte sorunlar yaşanabilir. Nüfus artışına yol açan en önemli etmenlerden biri ise göçlerdir.Özellikle büyük şehirlerde nüfus artışına neden olan göçler işsizlik, eğitim, barınma, çevre ve alt yapı sorunlarıyla beraber sağlık sorunlarını da beraberinde getirmektedir. Göç alan bölgelerde yeterli sağlık kuruluşunun olmaması, göçmenlerin düşük gelir düzeyi, sağlık sigortalarının olmaması, dil sorunu, aynı hanede çok sayıda ailenin yaşaması, yaşanan olağanüstü duruma ve şiddete bağlı gelişen ruhsal bozukluklar, beslenme ve hijyen sorunları, temiz içme suyu sağlanamaması ve atıkların uzaklaştırılamaması gibi faktörler de göç eden bireylerin sağlık koşullarını olumsuz yönde etkiler.Tüm olumsuz faktörler işsizlik ve yoksullukla beraber hastalık patlamasına neden olmaktadır ve günümüzde sorun olmayan hastalıklar yeniden sorun olmaya başlamaktadır. (\protect\hyperlink{ref-aydougan2017turkiye}{Aydoğan ve Metintaş, 2017}) (\protect\hyperlink{ref-fofana2023cross}{\textbf{fofana2023cross?}}) Nüfusu aldığı göçlerle beraber yoğunlaşan illerde ülkeye giriş ve çıkışlarda gerekli sağlık taramaları yapılmalı yayılma riski olan hastalıkların tedavisi sağlamdıktan sonra giriş ve çıkışlara izin verilmedir.Bunun için göçün olduğu illere gerekli sağlık kuruluşları kurulmalı ve denetlemeleri sağlanmalıdır.

\#\#\#Ekonomi
Çağdaş göç teorilerinin başlangıcı olarak görülen neo-klasik yaklaşımın merkezinde de itme-çekme yaklaşımı yer almaktadır. Neo-klasik yaklaşımda göç mikro ve makro düzeyde açıklanmıştır. Makro düzeyde göç, üretim faktörlerinin ülkeler arası dağılımındaki eşitsizlikten kaynaklanmakta ve göç hareketleri emeğin bol ve dolayısıyla ücretlerin düşük olduğu ülkelerden emeğin kıt ve ücretlerin yüksek olduğu ülkelere doğru gerçekleşmektedir.Mikro düzeyde ise göç, göç etme ve etmemenin avantaj ve dezavantajlarına ilişkin tam bilgiye sahip rasyonel bireyler tarafından verilen kararlarının sonucu gerçekleşmektedir.Özetle fayda maksimizasyonu, rasyonel seçim, bölge ve ülkeler arası faktör-fiyat farklılıkları ve işgücü hareketliliği ilkelerine dayanan neo-klasik ekonomi paradigmasını yansıtan itme-çekme modellerine göre, göçün başlıca nedeni ekonomiktir. {[}leblang2023ties{]} Hayatın yürütülmesini sağlayan temel konu ekonomidir.Hayat şartları ekonomik düzeyin izin verdiği ölçüde iyileştirilebilir.Verilen dış göçün önüne geçmek için sağlanması gereken şart hayat şartlarını iyileştirmeye yöelik bir ekonominin sağlanmış olmasıdır.Bu temelde bireylere sağlanan iş imkanları ile berbaber ülkede yürütülen ekonomi politikalarının bireylere olumlu yansımasını gerektirir.Bu sebeple göç veren illerin göç alan illerle karşılaştırılarak şartların iyileştirilmesi sağlanmalıdır.Nüfusun ülke genelide eşit dağılımını sağlayacak bir ekonomi politikası yürütülmelidir.

\newpage

\hypertarget{references}{%
\section{Kaynakça}\label{references}}

\hypertarget{refs}{}
\begin{CSLReferences}{1}{0}
\leavevmode\vadjust pre{\hypertarget{ref-aydougan2017turkiye}{}}%
Aydoğan, S. ve Metintaş, S. (2017). T{Ü}RK{İ}YE'YE GELEN DI{Ş} G{Ö}{Ç} VE SA{Ğ}LI{Ğ}A ETK{İ}LER{İ}. \emph{EST{Ü}DAM Halk Sa{ğ}l{ı}{ğ}{ı} Dergisi}, \emph{2}(2), 37-45.

\leavevmode\vadjust pre{\hypertarget{ref-csimcsek2018turkiye}{}}%
ŞİMŞEK, H. ve KULA, S. S. (2018). T{Ü}RK{İ}YE'N{İ}N G{Ö}{Ç}MEN POL{İ}T{İ}KASINDA {İ}HMAL ED{İ}LEN BOYUT: E{Ğ}{İ}TSEL UYUM PROGRAMI. \emph{Ahi Evran {Ü}niversitesi {İ}ktisadi ve {İ}dari Bilimler Fak{ü}ltesi Dergisi}, \emph{2}(2), 5-21.

\end{CSLReferences}

\end{document}
